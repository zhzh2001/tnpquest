\documentclass[hyperref,UTF8,a4paper]{ctexart}

\usepackage{amsmath,makecell,multirow,amsfonts,siunitx,amssymb}

%\usepackage{array}
%\newcolumntype{P}[1]{>{\centering\arraybackslash}p{#1}}

\usepackage{geometry}
\geometry{left=1in,right=1in,top=1in,bottom=1in}

\usepackage{longtable,booktabs,graphicx,float,adjustbox,xurl}
\usepackage[normalem]{ulem}
\providecommand{\tightlist}{%
  \setlength{\itemsep}{0pt}\setlength{\parskip}{0pt}}
  
\usepackage{fancyvrb,fvextra,color}
\DefineVerbatimEnvironment{Highlighting}{Verbatim}{breaklines,commandchars=\\\{\}}
% Add ',fontsize=\small' for more characters per line
\newenvironment{Shaded}{}{}
\newcommand{\AlertTok}[1]{\textcolor[rgb]{1.00,0.00,0.00}{\textbf{#1}}}
\newcommand{\AnnotationTok}[1]{\textcolor[rgb]{0.38,0.63,0.69}{\textbf{\textit{#1}}}}
\newcommand{\AttributeTok}[1]{\textcolor[rgb]{0.49,0.56,0.16}{#1}}
\newcommand{\BaseNTok}[1]{\textcolor[rgb]{0.25,0.63,0.44}{#1}}
\newcommand{\BuiltInTok}[1]{#1}
\newcommand{\CharTok}[1]{\textcolor[rgb]{0.25,0.44,0.63}{#1}}
\newcommand{\CommentTok}[1]{\textcolor[rgb]{0.38,0.63,0.69}{\textit{#1}}}
\newcommand{\CommentVarTok}[1]{\textcolor[rgb]{0.38,0.63,0.69}{\textbf{\textit{#1}}}}
\newcommand{\ConstantTok}[1]{\textcolor[rgb]{0.53,0.00,0.00}{#1}}
\newcommand{\ControlFlowTok}[1]{\textcolor[rgb]{0.00,0.44,0.13}{\textbf{#1}}}
\newcommand{\DataTypeTok}[1]{\textcolor[rgb]{0.56,0.13,0.00}{#1}}
\newcommand{\DecValTok}[1]{\textcolor[rgb]{0.25,0.63,0.44}{#1}}
\newcommand{\DocumentationTok}[1]{\textcolor[rgb]{0.73,0.13,0.13}{\textit{#1}}}
\newcommand{\ErrorTok}[1]{\textcolor[rgb]{1.00,0.00,0.00}{\textbf{#1}}}
\newcommand{\ExtensionTok}[1]{#1}
\newcommand{\FloatTok}[1]{\textcolor[rgb]{0.25,0.63,0.44}{#1}}
\newcommand{\FunctionTok}[1]{\textcolor[rgb]{0.02,0.16,0.49}{#1}}
\newcommand{\ImportTok}[1]{#1}
\newcommand{\InformationTok}[1]{\textcolor[rgb]{0.38,0.63,0.69}{\textbf{\textit{#1}}}}
\newcommand{\KeywordTok}[1]{\textcolor[rgb]{0.00,0.44,0.13}{\textbf{#1}}}
\newcommand{\NormalTok}[1]{#1}
\newcommand{\OperatorTok}[1]{\textcolor[rgb]{0.40,0.40,0.40}{#1}}
\newcommand{\OtherTok}[1]{\textcolor[rgb]{0.00,0.44,0.13}{#1}}
\newcommand{\PreprocessorTok}[1]{\textcolor[rgb]{0.74,0.48,0.00}{#1}}
\newcommand{\RegionMarkerTok}[1]{#1}
\newcommand{\SpecialCharTok}[1]{\textcolor[rgb]{0.25,0.44,0.63}{#1}}
\newcommand{\SpecialStringTok}[1]{\textcolor[rgb]{0.73,0.40,0.53}{#1}}
\newcommand{\StringTok}[1]{\textcolor[rgb]{0.25,0.44,0.63}{#1}}
\newcommand{\VariableTok}[1]{\textcolor[rgb]{0.10,0.09,0.49}{#1}}
\newcommand{\VerbatimStringTok}[1]{\textcolor[rgb]{0.25,0.44,0.63}{#1}}
\newcommand{\WarningTok}[1]{\textcolor[rgb]{0.38,0.63,0.69}{\textbf{\textit{#1}}}}

%\usepackage{chngcntr}
%\counterwithin{figure}{section}
%\counterwithin{table}{section}
%\renewcommand\thefigure{\thesection-\arabic{figure}}
%\renewcommand\thetable{\thesection-\arabic{table}}

\newcommand{\includegraphicx}[1]{\maxsizebox{\textwidth}{\textheight}{\includegraphics{#1}}}

\title{基于《猫武士·新预言》小说原著的MUD游戏设计\\
\bigskip \large 面向对象程序设计M3报告}
\author{nullptr组}
\date{2022 年 5 月 20 日}

\begin{document}
\maketitle

\hypertarget{ux4ebaux5458ux4fe1ux606f}{%
\section{人员信息}\label{ux4ebaux5458ux4fe1ux606f}}

组名:nullptr

组员:\textbf{略}

\hypertarget{ux672cux6708ux5177ux4f53ux8fdbux5ea6}{%
\section{本月具体进度}\label{ux672cux6708ux5177ux4f53ux8fdbux5ea6}}

\hypertarget{ux5b8cux6210ux6e38ux620fux4e3bux4f53ux4ee3ux7801ux7684ux64b0ux5199}{%
\subsection{完成游戏主体代码的撰写}\label{ux5b8cux6210ux6e38ux620fux4e3bux4f53ux4ee3ux7801ux7684ux64b0ux5199}}

在本周期的工作中,我们基本完成了游戏代码的撰写,使得游戏的主体部分得以正常运行,游戏基本的移动、战斗、饮水、触发随机事件等功能均已完成。现将游戏这些部分的代码做如下介绍:

\begin{Shaded}
\begin{Highlighting}[]
\CommentTok{//game.cpp}
\CommentTok{//pseudo code}

\DataTypeTok{bool}\NormalTok{ Game::process()}
\NormalTok{\{}
    \ControlFlowTok{if}\NormalTok{ (can }\ControlFlowTok{continue}\NormalTok{ to move) then}
        \ControlFlowTok{if}\NormalTok{ (input == }\CharTok{'q'}\NormalTok{) then}
\NormalTok{            quit;}
        \ControlFlowTok{if}\NormalTok{ (input == }\CharTok{'}\SpecialCharTok{\textbackslash{}n}\CharTok{'}\NormalTok{ || input == }\CharTok{'}\SpecialCharTok{\textbackslash{}r}\CharTok{'}\NormalTok{) then}
\NormalTok{            stop, lose val }\KeywordTok{and}\NormalTok{ go next;}
\NormalTok{        backup current coordinates;}
        \ControlFlowTok{if}\NormalTok{ (illegal move) then}
\NormalTok{            go next;}
        \ControlFlowTok{if}\NormalTok{ (can move diagonally) then}
\NormalTok{            put terminal into halfdelay mode;}
            \ControlFlowTok{if}\NormalTok{ (input == }\CharTok{'q'}\NormalTok{) then}
\NormalTok{                quit;}
            \ControlFlowTok{switch}\NormalTok{ terminal into normal mode;}
            \ControlFlowTok{if}\NormalTok{ (illegal move) then}
\NormalTok{                rollback }\KeywordTok{and}\NormalTok{ go next;}
        \ControlFlowTok{if}\NormalTok{ (trigger event) then}
\NormalTok{            generate step, lose val }\KeywordTok{and}\NormalTok{ go next;}
        \ControlFlowTok{if}\NormalTok{ (step <- step - }\DecValTok{1} \KeywordTok{and}\NormalTok{ step == }\DecValTok{0}\NormalTok{) then}
\NormalTok{            lose val;}
    \ControlFlowTok{else}\NormalTok{ then}
        \ControlFlowTok{do}\NormalTok{ action}
        \ControlFlowTok{if}\NormalTok{ (quit action) then}
\NormalTok{            quit;}
\NormalTok{        generate step;}
\NormalTok{    go next;}
\NormalTok{\}}
\end{Highlighting}
\end{Shaded}

\texttt{process()}
函数主要负责处理猫的移动。判断当前状态是否可以继续移动,如果可以则从键盘读入输入,然后执行移动命令。其中影族可以斜着走是通过短时间内同时按下两个方向键完成的,在上次报告中已经详细介绍,这里不再赘述。若没有剩余步数,则扣除一定的属性值,玩家可以选择六种行动中的一种(见下文),并生成新一轮的移动步数。

\begin{Shaded}
\begin{Highlighting}[]
\CommentTok{//game.cpp}
\CommentTok{//pseudo code}

\DataTypeTok{void}\NormalTok{ Game::renderStatus()}
\NormalTok{\{}
    \ControlFlowTok{if}\NormalTok{ (fallback) then}
\NormalTok{        print the title of all status;}
\NormalTok{        print the frame of all status;}
    \ControlFlowTok{switch}\NormalTok{ (clan of cat)}
        \ControlFlowTok{case}\NormalTok{ thunder: print }\StringTok{"ThundrClan"}\NormalTok{;}
        \ControlFlowTok{case}\NormalTok{ wind: print }\StringTok{"WindClan"}\NormalTok{;}
        \ControlFlowTok{case}\NormalTok{ river: print }\StringTok{"RiverClan"}\NormalTok{;}
        \ControlFlowTok{case}\NormalTok{ shadow: print }\StringTok{"ShadowClan"}\NormalTok{;}
        \ControlFlowTok{default}\NormalTok{: print }\StringTok{"Unknown"}\NormalTok{;}
\NormalTok{    print the contents of other status;}
\NormalTok{    print the frame of all status;}
\NormalTok{\}}
\end{Highlighting}
\end{Shaded}

\texttt{renderStatus()}
函数主要负责渲染属性栏底下的信息框,即将猫的族群、学徒名、地图编号(原游戏没有)、剩余步数、已收集星数等信息展示出来。

\begin{Shaded}
\begin{Highlighting}[]
\CommentTok{//game.cpp}
\CommentTok{//pseudo code}

\DataTypeTok{bool}\NormalTok{ Game::doAction()}
\NormalTok{\{}
\NormalTok{    print the buttons of action;}
\NormalTok{    choice <- }\DecValTok{0}\NormalTok{;}
    \ControlFlowTok{for}\NormalTok{ (;;) }\ControlFlowTok{do}
        \ControlFlowTok{for}\NormalTok{ (i <- }\DecValTok{0}\NormalTok{ to }\DecValTok{2}\NormalTok{) }\ControlFlowTok{do}
            \ControlFlowTok{for}\NormalTok{ (j <- }\DecValTok{0}\NormalTok{ to }\DecValTok{3}\NormalTok{) }\ControlFlowTok{do}
                \ControlFlowTok{if}\NormalTok{ (i * }\DecValTok{3}\NormalTok{ + j == choice) then}
\NormalTok{                    turn on the A_REVERSE attributes;}
\NormalTok{                print the selected button in reverse color;}
\NormalTok{                turn off the A_REVERSE attributes;}
        \ControlFlowTok{if}\NormalTok{ (input == }\CharTok{'q'}\NormalTok{) then}
\NormalTok{            quit;}
        \ControlFlowTok{if}\NormalTok{ (input == }\CharTok{'}\SpecialCharTok{\textbackslash{}n}\CharTok{'}\NormalTok{ || input == }\CharTok{'}\SpecialCharTok{\textbackslash{}r}\CharTok{'}\NormalTok{) then}
            \ControlFlowTok{break}\NormalTok{ from the loop;}
        \ControlFlowTok{if}\NormalTok{ (input == KEY_UP) then}
\NormalTok{            move up the choice;}
        \ControlFlowTok{else} \ControlFlowTok{if}\NormalTok{ (ch == KEY_DOWN) then}
\NormalTok{            move down the choice;}
        \ControlFlowTok{else} \ControlFlowTok{if}\NormalTok{ (ch == KEY_LEFT) then}
\NormalTok{            move left the choice;}
        \ControlFlowTok{else} \ControlFlowTok{if}\NormalTok{ (ch == KEY_RIGHT) then}
\NormalTok{            move right the choice;}
\NormalTok{        print the blank;}
\NormalTok{        refresh the screen;}
    \ControlFlowTok{switch}\NormalTok{ (choice)}
        \ControlFlowTok{case} \DecValTok{0}\NormalTok{: go battle;}
        \ControlFlowTok{case} \DecValTok{1}\NormalTok{: go check scent;}
        \ControlFlowTok{case} \DecValTok{2}\NormalTok{: go hunt;}
        \ControlFlowTok{case} \DecValTok{3}\NormalTok{: go explore;}
        \ControlFlowTok{case} \DecValTok{4}\NormalTok{: go rest;}
        \ControlFlowTok{case} \DecValTok{5}\NormalTok{:}
        \ControlFlowTok{if}\NormalTok{ (no water) then}
\NormalTok{            MsgBox print }\StringTok{"No water here"}\NormalTok{;}
        \ControlFlowTok{else}\NormalTok{ then}
\NormalTok{            go drink;}
\NormalTok{    go next;}
\NormalTok{\}}
\end{Highlighting}
\end{Shaded}

\texttt{doAction()}
函数主要负责对猫的行动进行管理。在控制台的底部,我们会打印出战斗、检查气味、捕猎、探索、休息、喝水六个按钮。玩家可以在六个按钮中选择一个进行行动,而我们的
\texttt{doAction()}
函数则会给出响应的控制。同时,和地图移动一样,我们对当前选择的按钮也进行了反色处理,使得玩家可以清楚看出自己目前位于哪个按钮上。

为了有关类调用方便,战斗、检查气味、捕猎和探索由 \texttt{Board}
类完成,休息和喝水由 \texttt{Cat}
类完成,因为前面几个活动和地图数据密切相关。事实上,战斗可以直接调用
\texttt{Enemy::trigger}
来完成,而检查气味和探索完全是地图操作。捕猎比较复杂,如果玩家直接走到猎物的格子上,会触发
\texttt{Prey::trigger} 而吓走猎物,但如果主动捕猎则可能能捕捉到猎物。

\hypertarget{ux5bf9ux7c7bux529fux80fdux7684ux5b9eux73b0ux4e0eux5b8cux5584}{%
\subsection{对类功能的实现与完善}\label{ux5bf9ux7c7bux529fux80fdux7684ux5b9eux73b0ux4e0eux5b8cux5584}}

在本月的工作进程中,我们对此前已经定义、但还未完成功能的类进行了功能的填充与改善,目前我们游戏正常运行所需的类功能基本已经齐全。现将这些部分的代码做如下介绍:

\begin{Shaded}
\begin{Highlighting}[]
\CommentTok{// cat.h}

\DataTypeTok{int}\NormalTok{ Cat::generateStep() }\AttributeTok{const}\NormalTok{;}
\CommentTok{// generate step from 1 to speed;if clan is wind, add 2 to the step}

\DataTypeTok{void}\NormalTok{ Cat::setMaxStep(}\DataTypeTok{int}\NormalTok{ m);}
\CommentTok{// set maxStep = m}

\DataTypeTok{void}\NormalTok{ Cat::loseVal();}
\CommentTok{// if no more step, lose health, hunger and thirst}

\DataTypeTok{void}\NormalTok{ Cat::checkDead() }\AttributeTok{const}\NormalTok{;}
\CommentTok{// if not invincible and any status < 0, game over}

\DataTypeTok{void}\NormalTok{ Cat::loseHealth(}\DataTypeTok{int}\NormalTok{ h);}
\CommentTok{// set health -= h}

\DataTypeTok{void}\NormalTok{ Cat::gainHealth(}\DataTypeTok{int}\NormalTok{ h);}
\CommentTok{// set health += h}

\DataTypeTok{void}\NormalTok{ Cat::gainHunger(}\DataTypeTok{int}\NormalTok{ h);}
\CommentTok{// set hunger += h}

\DataTypeTok{void}\NormalTok{ Cat::gainThirst(}\DataTypeTok{int}\NormalTok{ h);}
\CommentTok{// set thirst += h}

\DataTypeTok{int}\NormalTok{ Cat::rollVal(CatVal val) }\AttributeTok{const}\NormalTok{;}
\CommentTok{// roll val from 1 to skill/speed}

\DataTypeTok{int}\NormalTok{ Cat::rollAttack() }\AttributeTok{const}\NormalTok{;}
\CommentTok{// roll attack from 1 to strength;if clan is thunder, add 2 to the attack}

\DataTypeTok{int}\NormalTok{ Cat::rollDefense() }\AttributeTok{const}\NormalTok{;}
\CommentTok{// roll defense from 1 to defense}

\DataTypeTok{void}\NormalTok{ Cat::flee();}
\CommentTok{// decrease health, hunger, thirst to flee from battle}

\DataTypeTok{bool}\NormalTok{ Cat::hasFled();}
\CommentTok{// return fled}

\DataTypeTok{void}\NormalTok{ Cat::resetFled();}
\CommentTok{// reset fled to false}

\DataTypeTok{void}\NormalTok{ Cat::incStar();}
\CommentTok{// increase stars}

\DataTypeTok{int}\NormalTok{ Cat::getStars() }\AttributeTok{const}\NormalTok{;}
\DataTypeTok{bool}\NormalTok{ Cat::getDiagonal() }\AttributeTok{const}\NormalTok{;}
\DataTypeTok{bool}\NormalTok{ Cat::getWaterproof() }\AttributeTok{const}\NormalTok{;}
\CommentTok{// public method to get the value of star, diagonal and waterproof}

\DataTypeTok{void}\NormalTok{ Cat::rest();}
\CommentTok{// roll to gain 1 or 2 health, show the info in msgbox}

\DataTypeTok{void}\NormalTok{ Cat::drink();}
\CommentTok{// roll to gain 1 or 2 health, show the info in msgbox}
\end{Highlighting}
\end{Shaded}

\texttt{Cat}
类新增了有关移动、喝水、休息相关的公有函数成员,方便实现游戏中的相关功能。同时还增加了显示及修改stars,
diagonal,
waterproof等属性的公开方法,方便在类外对这些私有成员进行读取或修改。

\begin{Shaded}
\begin{Highlighting}[]
\CommentTok{// board.cpp}

\DataTypeTok{void}\NormalTok{ Board::backupCoordinates();}
\CommentTok{// record current position into backup ones}

\DataTypeTok{void}\NormalTok{ Board::restoreCoordinates();}
\CommentTok{// set current position with backup ones}

\DataTypeTok{bool}\NormalTok{ Board::trigger(Cat &cat);}
\CommentTok{// trigger events and check if fled}

\DataTypeTok{int}\NormalTok{ Board::getScreen() }\AttributeTok{const}\NormalTok{;}
\CommentTok{// public method to get the value of nowscreen}

\DataTypeTok{void}\NormalTok{ Board::moveTarget(}\DataTypeTok{int}\NormalTok{ ch);}
\CommentTok{// modify the target row and col}

\DataTypeTok{void}\NormalTok{ Board::chooseTarget(}\DataTypeTok{bool}\NormalTok{ diagonal, }\DataTypeTok{bool}\NormalTok{ fallback);}
\CommentTok{// check keyboard input to choose target, print the target cell in reverse color}

\DataTypeTok{void}\NormalTok{ Board::battle(Cat &cat, }\DataTypeTok{bool}\NormalTok{ fallback);}
\CommentTok{// check if there is an enermy; then check if fled and print the info in msgbox}

\DataTypeTok{void}\NormalTok{ Board::hunt(Cat &cat, }\DataTypeTok{bool}\NormalTok{ fallback);}
\CommentTok{// check if can hunt and record current position}

\DataTypeTok{void}\NormalTok{ Board::explore(Cat &cat, }\DataTypeTok{bool}\NormalTok{ fallback);}
\CommentTok{// explore and set the visibility}

\DataTypeTok{bool}\NormalTok{ Board::getWater() }\AttributeTok{const}\NormalTok{;}
\CommentTok{// check if there has water}
\end{Highlighting}
\end{Shaded}

\texttt{Board}
类保存地图相关的信息。在本月的进度中,我们设置的方法主要是为了控制当前位置,以及对地图的周围格子进行战斗、捕猎、探索、喝水等操作。

\begin{Shaded}
\begin{Highlighting}[]
\CommentTok{// cell.cpp}

\DataTypeTok{bool}\NormalTok{ Cell::trigger(Cat &cat);}
\CommentTok{// check if there is an enermy to battle and deal with fled}

\DataTypeTok{bool}\NormalTok{ Cell::isEnemy() }\AttributeTok{const}\NormalTok{;}
\CommentTok{// check if item is enermy}

\DataTypeTok{bool}\NormalTok{ Cell::hunt(Cat &cat);}
\CommentTok{// check if item is prey}

\AttributeTok{const}\NormalTok{ Terrain *Cell::getTerrain() }\AttributeTok{const}\NormalTok{;}
\CommentTok{// return terrain}
\end{Highlighting}
\end{Shaded}

\texttt{Cell}
类表示地图上的一个格子,每个格子都有一个地形。在上个周期的工作中我们已经实现了这些,而在本周期的工作中,我们主要为该类添加了触发器以触发战斗事件、对当前格子是否有敌人的判断以及是否可以捕猎的判断;此外,还公开了获取地形属性的方法。

而在 \texttt{item.h} 中,我们在 \texttt{Item} 类中定义了一个纯虚函数
\texttt{trigger} 以供子类使用。而在继承自 \texttt{Item} 的
\texttt{Enemy}, \texttt{DefenseAction}, \texttt{Injury}, \texttt{Prey},
\texttt{Benefit} 几个类中,我们均对该纯虚函数进行了重载。接下来对这些
\texttt{trigger} 进行简单介绍:

\begin{Shaded}
\begin{Highlighting}[]
\CommentTok{//triggers from enemy.cpp, defense.cpp, injury.cpp, prey.cpp}
\CommentTok{//pseudo code}

\DataTypeTok{void}\NormalTok{ Enemy::trigger(Cat &cat) }\AttributeTok{const}
\NormalTok{\{}
\NormalTok{    roll attack }\KeywordTok{and}\NormalTok{ health of enemy;}
\NormalTok{    first <- }\KeywordTok{true}\NormalTok{;}
    \ControlFlowTok{do}
\NormalTok{        print attack info in msg;}
        \ControlFlowTok{if}\NormalTok{ (first) }\ControlFlowTok{do}
\NormalTok{            create msgbox;}
        \ControlFlowTok{else} \ControlFlowTok{do}
            \DataTypeTok{int}\NormalTok{ option <- defend }\KeywordTok{or}\NormalTok{ flee;}
            \ControlFlowTok{if}\NormalTok{ (option == }\DecValTok{2}\NormalTok{) }\ControlFlowTok{do}
\NormalTok{                flee }\KeywordTok{and} \ControlFlowTok{return}\NormalTok{;}
\NormalTok{        first <- }\KeywordTok{false}\NormalTok{;}
\NormalTok{        roll defense of cat;}
\NormalTok{        dmg_to_cat <- max\{}\DecValTok{0}\NormalTok{, attack_of_enemy - defense_of_cat\};}
\NormalTok{        cat lose health;}
\NormalTok{        roll attack of cat;}
\NormalTok{        roll defense of enemy from }\DecValTok{0}\NormalTok{ to defense;}
\NormalTok{        dmg_to_enemy <- max\{}\DecValTok{0}\NormalTok{, attack_of_cat - defense_of_enemy\};}
\NormalTok{        enemy lose health;}
\NormalTok{        print lose health info in msgbox;}
    \ControlFlowTok{while}\NormalTok{ (enemyHealth > }\DecValTok{0}\NormalTok{);}
\NormalTok{\}}

\DataTypeTok{void}\NormalTok{ DefenseAction::trigger(Cat &cat) }\AttributeTok{const}
\NormalTok{\{}
\NormalTok{    roll damage in damagerange;}
\NormalTok{    roll result;}
\NormalTok{    print defense info in msg;}
\NormalTok{    result <- max\{damage - result, }\DecValTok{0}\NormalTok{\};}
\NormalTok{    print health losing info in msg;}
\NormalTok{    create msgbox;}
\NormalTok{    cat lose health;}
\NormalTok{\}}

\DataTypeTok{void}\NormalTok{ Injury::trigger(Cat &cat) }\AttributeTok{const}
\NormalTok{\{}
\NormalTok{    roll damage from }\DecValTok{1}\NormalTok{ to damage;}
\NormalTok{    print max_damage }\KeywordTok{and}\NormalTok{ injury info in msgbox;}
\NormalTok{    cat lose health;}
\NormalTok{\}}

\DataTypeTok{void}\NormalTok{ Prey::trigger(Cat &cat) }\AttributeTok{const}
\NormalTok{\{}
\NormalTok{    print scared away info in msgbox;}
\NormalTok{\}}

\DataTypeTok{void}\NormalTok{ Benefit::trigger(Cat &cat) }\AttributeTok{const}
\NormalTok{\{}
\NormalTok{    roll dice to see whether to use the benefit}
    \ControlFlowTok{if}\NormalTok{ (dice > require)}
\NormalTok{        apply health/hunger/thirst to cat}
\NormalTok{        show success in msgbox}
    \ControlFlowTok{else}
\NormalTok{        show fail in msgbox}
\NormalTok{\}}
\end{Highlighting}
\end{Shaded}

从上数代码中我们可以看出,无论是遇敌战斗的事件、在战斗中防御的事件、遭遇受伤的事件,还是吓走猎物的事件,其触发器都是基于
\texttt{Item} 类的纯虚函数 \texttt{trigger}
中实现的。这使得我们的程序实现了动态联编,实现了不同触发器之间接口的统一。这样当玩家在地图上移动时,遇到各种物品,只需要调用
\texttt{trigger} 即可实现对应的触发事件。

\hypertarget{ux8fd0ux884cux622aux56fe}{%
\section{运行截图}\label{ux8fd0ux884cux622aux56fe}}

\begin{figure}[H]
\centering
\includegraphicx{img/battle.png}
\caption{battle}
\end{figure}

\begin{figure}[H]
\centering
\includegraphicx{img/flee.png}
\caption{flee}
\end{figure}

\begin{figure}[H]
\centering
\includegraphicx{img/gameover.png}
\caption{game over}
\end{figure}

\begin{figure}[H]
\centering
\includegraphicx{img/rest.png}
\caption{rest}
\end{figure}

\begin{figure}[H]
\centering
\includegraphicx{img/hunt.png}
\caption{hunt}
\end{figure}

\begin{figure}[H]
\centering
\includegraphicx{img/explore.png}
\caption{explore}
\end{figure}

\begin{figure}[H]
\centering
\includegraphicx{img/drink.png}
\caption{drink}
\end{figure}

\begin{figure}[H]
\centering
\includegraphicx{img/cnnew.png}
\caption{中文界面}
\end{figure}

其中中文界面在 Windows 下仍有很多对齐的 bug,主要是 Windows 终端对
Unicode 支持实在是不佳,可能不会修复。英文无影响。

\hypertarget{ux4e0bux5468ux671fux5de5ux4f5cux8ba1ux5212}{%
\section{下周期工作计划}\label{ux4e0bux5468ux671fux5de5ux4f5cux8ba1ux5212}}

\begin{itemize}
\tightlist
\item
  继续针对代码细节进行debug,完善有关违规操作的处理方式
\item
  进行工程整体的收尾工作,撰写报告,完善代码注释等
\end{itemize}
\end{document}